\documentclass[letterpaper, 10 pt, conference]{ieeeconf}  % Comment this line out
                                                          % if you need a4paper
%\documentclass[a4paper, 10pt, conference]{ieeeconf}      % Use this line for a4
                                                          % paper

\IEEEoverridecommandlockouts                              % This command is only
                                                          % needed if you want to
                                                          % use the \thanks command
\overrideIEEEmargins
% See the \addtolength command later in the file to balance the column lengths
% on the last page of the document

% Add additional packages here if required
\usepackage{color}
\definecolor{highlight}{rgb}{1,1,0.6}
\definecolor{link}{rgb}{0.5,0.0,0.0}
\definecolor{cite}{rgb}{0.0,0.0,0.6}
\definecolor{url} {rgb}{0.3,0.0,0.3}
\definecolor{grey}{rgb}{0.3,0.3,0.3}
\usepackage{siunitx}
\usepackage{adjustbox} % by awa
%***
\usepackage{listings}
\usepackage{url}
\usepackage{array, makecell} %

\usepackage[linesnumbered,ruled]{algorithm2e} % used for algorithms
\newcommand\mycommfont[1]{\footnotesize\ttfamily\textcolor{blue}{#1}}
\SetCommentSty{mycommfont}
%***
\usepackage{amsmath}
\usepackage{graphicx}
\usepackage{fancybox}
\usepackage{rotating} % to rotate the table
\usepackage{multirow} % used to merger columns in table
\usepackage{soul} % for highlighting text
\usepackage{relsize} % used in the \anote and \comment macros.
\usepackage{subfig} % used to put figure side by side
\usepackage{mathtools}
\DeclarePairedDelimiter{\ceil}{\lceil}{\rceil}

\newcommand{\quotes}[1]{``#1''} 
\makeatletter
\newcommand{\mathleft}{\@fleqntrue\@mathmargin15pt}
\newcommand{\mathcenter}{\@fleqnfalse}
\makeatother

\newcommand{\mynote}[1]{{\leavevmode\smaller\itshape\color{red}\{#1\}}}
\sethlcolor{highlight}
\newcommand{\mycomment}[2]{\hl{#1} {{\leavevmode\smaller\color{red}\itshape\{#2\}}}}

\newcommand{\anote}[1]{{\leavevmode\smaller\itshape\color{red}\{#1\}}}

\newcommand{\topic}{\ensuremath{\mathit{T}}}
\newcommand{\bs}{\ensuremath{\mathit{BS}}}
\newcommand{\ti}{\ensuremath{\mathit{TI}}}
\newcommand{\mr}{\ensuremath{\mathit{MR}}}
\newcommand{\up}{\ensuremath{\mathit{UP}}}
\newcommand{\doff}{\ensuremath{\mathit{DO}}}
\newcommand{\tati}{\ensuremath{\mathit{TATI}}}
\newcommand{\etm}{\ensuremath{\mathit{ETM}}}
\newcommand{\tp}{\ensuremath{\mathit{TP}}}
\newcommand{\bn}{\ensuremath{\mathit{BN}}}
\newcommand{\smartc}{\ensuremath{\mathit{SC}}}
\newcommand{\bt}{\ensuremath{\mathit{BT}}}
\newcommand{\rtEst}{\ensuremath{\widehat{\mathit{RT}}}}

\newcommand{\crep}{\ensuremath{C_{\mathit{rep}} }}

\newcommand{\gt}{\ensuremath{\mathit{GT}}}
\newcommand{\rc}{\ensuremath{\mathit{RC}}}   % receipt cost
\newcommand{\gup}{\ensuremath{\mathit{GUP}}}
%
% The following packages can be found on http:\\www.ctan.org
\usepackage{graphics} % for pdf, bitmapped graphics files
%\usepackage{epsfig} % for postscript graphics files
%\usepackage{mathptmx} % assumes new font selection scheme installed
%\usepackage{times} % assumes new font selection scheme installed
\usepackage{amsmath} % assumes amsmath package installed
\usepackage{amssymb}  % assumes amsmath package installed

\title{\LARGE \bf
A Decentralized, Trust-less Marketplace for Brokered IoT Data Trading using Blockchain
}

%\author{ \parbox{3 in}{\centering Huibert Kwakernaak*
%         \thanks{*Use the $\backslash$thanks command to put information here}\\
%         Faculty of Electrical Engineering, Mathematics and Computer Science\\
%         University of Twente\\
%         7500 AE Enschede, The Netherlands\\
%         {\tt\small h.kwakernaak@autsubmit.com}}
%         \hspace*{ 0.5 in}
%         \parbox{3 in}{ \centering Pradeep Misra**
%         \thanks{**The footnote marks may be inserted manually}\\
%        Department of Electrical Engineering \\
%         Wright State University\\
%         Dayton, OH 45435, USA\\
%         {\tt\small pmisra@cs.wright.edu}}
%}

\author{Shaimaa Bajoudah$^{1,2}$, Changyu Dong$^{1}$ and Paolo Missier$^{1}$% <-this % stops a space
% \thanks{*This work was not supported by any organization}% <-this % stops a space
\thanks{$^{1}$School of Computing, Newcastle University, Newcastle Upon Tyne, The UK.
         Email: s.bajoudah1@ncl.ac.uk, Changyu.Dong@ncl.ac.uk and Paolo.Missier@ncl.ac.uk}%
\thanks{$^{2}$College of Computer and Information Systems, Umm AlQura University, Makkah, KSA. Email: sbajoudah@uqu.edu.sa}%
}


\begin{document}


\maketitle
\thispagestyle{empty}
\pagestyle{empty}


%%%%%%%%%%%%%%%%%%%%%%%%%%%%%%%%%%%%%%%%%%%%%%%%%%%%%%%%%%%%%%%%%%%%%%%%%%%%%%%%
\begin{abstract}

As data marketplaces are becoming commonplace, it is also becoming clear that data streams generated from Internet of Things (IoT) devices hold value for potential third party consumers.
We envision a marketplace for IoT data streams that can unlock such potential value in a scalable way, by enabling any pairs of data providers and consumers to engage in data exchange transactions without any prior assumption of mutual trust.
We present a marketplace model and architecture to support trading of streaming data, from the advertising of data assets to the stipulation of legally binding trading agreements, to their fulfilment and payment settlement. Crucially, we show that by using blockchain technology and Smart Contracts in particular, we can offer participants a trade-off between the cost of transactional data exchange, and the risk of data loss when trading with untrusted third parties.
We experimentally assess such trade-offs on a testbed using Ethereum Smart Contracts.
%
%data are increasingly viewed as a new form of massively distributed and large scale digital assets, which are continuously generated by millions of connected devices. The real value of such assets can only be realized The IoT data that are generated from IoT devices has shown a new insight to do profitable business. Any IoT data owner industries or individuals can sell their data into a marketplaces and get some profits. This paper has introduced a decentralized marketplace for real-time IoT data streams with the use of the Blockchain technology. A producer publishes offers to what streams has, and a consumer request a trade based on the proposed offers. Then a data producer and data consumer has a trade agreement  TA for their trade recorded in the Blockchain and executed if the contractual obligation has been met. An honest consumer who send receipts with the exact amount of data received. It shows a direct proportion between a consumer honesty and trade's cost done in this marketplace. It has been evaluated that an honest consumer cost less than dishonest consumer to get involved in trades in this marketplace. 

\end{abstract}


%%%%%%%%%%%%%%%%%%%%%%%%%%%%%%%%%%%%%%%%%%%%%%%%%%%%%%%%%%%%%%%%%%%%%%%%%%%%%%%%
\section{Introduction}  \label{sec:intro}

Data streams that originate from Internet of Things (IoT) devices are increasingly viewed as tradeable assets with value not only to the device owners, but also with resell value, i.e., to third party buyers. 
New forms of dedicated data marketplaces are emerging to help unlock such value~\cite{misura}, but these are comparatively less mature than more traditional data marketplaces for static data, cf. eg ~\cite{7004800,Schomm2013} for surveys on these.
Unlike static data,  IoT data streams tend to lose their value if they are not consumed in near-real time, and data transmission and delivery may be unreliable. 
On the other hand, data exchange architectures based on message brokers systems such as MQTT allow a single data stream to be delivered to multiple parties, potentially enabling large-scale  open marketplaces where data owners may resell their streams in realt-time multiple times.
While the IoT network and message-passing infrastructure can support a scalable marketplace, this inevitably leads to issues of mutual trust amongst participants, especially when those have no prior reputation within the marketplace. Also, the short-lived nature of streams requires efficient, automated mechanisms to create legally binding trade agreements, including payment arrangements, and to enforce such agreements throughout data transmission.

New generation blockchain technology that supports Smart Contracts is a natural choice to address all of these requirements, as Smart Contracts can act as a trusted intermediary within an untrusted community of marketplace participants, by adding transactionality to each of their interactions: before, during, and after data exchange.

An example of such approach is Datum~\cite{Haenni2017} (\url{datuk.org}), based on the Ethereum network, which however is designed to let anyone \textit{store} structured data on the blockchain.
In contrast, we envision a decentralised marketplace for real-time IoT data that is scalable in the number of participants and does not require prior trust amongst them, while at the same time providing simple guarantees regarding data and monetary loss in case of participant's fraud.
The marketplace should be able to flexibly accept new participants (either individuals, institutions or business organizations), be resilient to leaving participants, and accommodate unanticipated business relationships amongst those participants. That means anyone who controls IoT devices and generate IoT data streams should be able to monetize it and use it as a tradeable assets in the marketplace.
Additionally, in contrast to existing proposals, e.g.\cite{Cao:2016:MMR:2926746.2883611}, we aim to define a marketplace that does not require a centralized trust component, such as a brokerage platform with trusted ownership, but relies instead on collective verification mechanisms, such as blockchain, to enforce its own governance rules.

Our approach involves using Ethereum Smart Contracts to support each phase of the interaction amongst a data provider and a consumer. It separates the data exchange interaction, which occurs on the IoT network  and core cloud network, from transaction-based interactions aimed at enforcing non-repudiability of participant's actions and resolving their disputes, which occurs on the blockchsin network.

\subsection{Contributions} \label{sec:contributions}
%%%%%%%%%%%%%%%%%%%%%%%%%%%%%%

This work follows on from, and refines, our earlier proposal for a IoT data marketplace, where we suggested that Ethereum is capable of supporting a fully decentralised marketplace without any assumption of mutual trust~~\cite{Missier2017}.
The approach suggested in the paper is based on the idea that each participant would periodically report to a Smart Contract on the data sent to and received from other participants, and the Contract would be able to use such reports to settle any disputes.

In contrast, here we begin by proposing a different and much simpler protocol involving data providers, consumers, and a Smart Contract, based on the notion of periodic checkpoints during data exchange, supported by blockchain transactions to ensure limited scope for fraud on either side. 

We then use our own prototype implementation of the  marketplace model on a private Ethereum network, to experimentally evaluate the cost/risk trade-offs that are available by setting checkpoint frequency, also taking advantage of \textit{potential} external mechanisms for establishing trust amongst participants, \textit{if they are} available.

% In addition, a blockchain is not designed to record too much data, so it has to employ the advantages of the blockchain in the marketplace for trades' obligations fulfillment and data exchange monitoring in the brokerage platform with taking into account the blockchain limitation e.g. scalability.

% The emerging of the decentralized applications has emphasized the lack of reliance on the third party and killed the fact of trusting an individual node.  It shows a wide area for applications with no failure point due to decentralized infrastructure. 

%  In our decentralized model, the settlement and the trade monitoring is done on the blockchain layer which means that we need a level of trust to accomplish theses trades successfully. The blockchain gives the opportunity to remove the third party that guarantee the trust in the centralized model , and instead distribute the trust and placed elsewhere, namely in public key cryptography and a ''consensus mechanism'' that allows us to determine the truth.

% Real-time IoT data has minimal value when they are not used at the same time of its generation, so we need a model dealing immediately with the data. In addition, the blockchain was not emerged to store a huge amount of data and instead, it is created to store limited data immutably. 

% With the core advantage of the second generation of the blockchain – smart contract – we propose each trade must have a digital contract called Trade Agreement TA. Each TA must be signed by both parties and once its obligations are met, it is executed automatically and all trades payments are done in Ethereum token (Ether).

% For this, unlike the IOTA project \cite{20}, there is no need to record all IoT data in the chain and only need to store the metadata and the trade details between a producer and a consumer as a trade agreement. With the core advantage of the second generation of the blockchain – smart contract – we propose each trade must have a digital contract called Trade Agreement TA. Each TA must be signed by both parties and once its obligations are met, it is executed automatically and all trades payments are done in Ethereum token (Ether).


%is increasingly viewed as is a dynamic global network for interconnected devices that can stream that are connecting together by the Internet such as sensors, actuators and Radio-frequency identification (RFIDs) to provide value-added services. Over the last decade, the IoT data has become one of the traceable assets in current ecosystem due to its beneficial aspects for their owners and their buyers. Data marketplaces \anote{for IoT data} have emerged which present a platform for data owners and buyers to negotiate and apply terms that both agreed to trade the needed data 

% Currently, there are many projects working as a marketplace for data trading. 
% \mycomment{Some of them}{need references} are showing a shared economy with real-time IoT data which generated from IoT devices in many domains. 
% \mycomment{Many of them}{need references} are offering various datasets to be sold for either individuals, government or industries for the aggregation and analysis purposes. 

% These traditional marketplaces has been selling the datasets where the data are not updated or not time sensitive. 
% Unlike the marketplaces for real-time data that are generated from IoT. 
% It is a difficult task to  design marketplaces for IoT data due to its characteristics which make them different from traditional datasets. \cite{misura} has shown unique IoT data features. The undependability of IoT data owner is one of its uniqueness. There are variety of IoT owner who produce data from their owned devices and sell it
%  In addition, IoT data are gathered from real-time sensors such as temperature, humidity or car traffic is time sensitive. 
%  If these data has not been used at the time of its generation it would loose their value and no longer needed. And because of the IoT data is time sensitive, a data buyer would have a trade agreement to buy data in advance which is a challenging task. Buying data in advance needs a mechanism in the marketplace to decide if the payment would be paid before of after delivering the data. If it is before subscribing the data, that means we require a system to fulfill the trade agreement that data provide and data buyer have, enforce the contract obligations to be done and guarantee delivering data to the consumers.

% Currently, it has sown many marketplaces that are tailored to deal with IoT data in either centralized or in decentralized infrastructure. Misura in \cite{misura} has introduced a centralized marketplace as a cloud application based on a queuing system. Each data provider registers their sensors into the application with all details and description of the data. On the other hand, data consumer query the system for their needs and then  they will be advised by the system to the appropriate data provider who provide their needs. 

% On the other hand, Datum \cite{24} has introduced a decentralized marketplace based on the emerging technology Blockchain. It is an IoT data marketplace which is designed to trade a stored IoT data based on Ethereum Blockchain network. A data provider achieves the scalability by storing his structural data securely in such BigchainDB and IPFS.  It uses the smart contract and the DAT token to enable the data buyer to get access to these data. 

% The vision of our model is to have a decentralized marketplace for real-time IoT data with the use of the distributed ledger (Blockchain). It consists of two layers; data layer for exchanging data in the brokerage platform and Blockchain layer for the negotiation and trades contracts.

% The remainder of this paper is organized as follows. Section \ref{CentModel} shows the centralized model. Section \ref{Motivation&Contribution} and \ref{RelatedWork&Background} show the motivation of our model and the related works. Section \ref{MKModel} and section \ref{Arch} show the marketplace model and the system architecture. Section \ref{Impl&Evaluation} shows the implementation and the evaluation of  consumer cost in the marketplace and then it will be ended by the conclusion in section \ref{conclusion}.


% In this paper we describe a Blockchain-based marketplace architecture that is designed to fulfill all of these 
\subsection{Related Work} \label{sec:related}

% Literately, the blockchain technology is not a new concept in the computer community, it is just the result of the combination of three old concepts, peer-to-peer networks, public key cryptography and distributed consensus based on resolution random math challenges (Cryptographic hashing) \cite{2}. It can be defined as a fully-distributed, peer to peer network platform which can record immutable data, host applications \cite{2} and even transfer the ownership of digital assets which has a monetary value in the reality. Although it currently comes at a great computational cost, the characteristic of not relying on a trusted third party is held as their most important characteristic.

% With the assumption of having a decentralized marketplace for IoT data leveraged by blockchain technology, we guarantee the trustworthy and transparency. The Immutability of transactions recorded in the blockchain is one of its strength and the key factor of providing a digital trustless environment to businesses. In the blockchain network, each transaction has been sent to the network is secure and transparent, thereby making the use of blockchain as a business infrastructure a good choice.

% Recently, many of financial sectors take the benefits of the blockchain (security, immutability, transparency) and trying to dispense with the intermediation and it will change the shape of companies’ friction everywhere \cite{2}\cite{9}\cite{10}. In the scenario of brokered IoT data marketplace, it is no longer need to pay fees for settlement provider to do the settlement in an untrusted environment. Moving to the decentralization infrastructure using blockchain solves one of critical issues in the in distributed environment as each node has to have the same version and the last update of the records

% The second generation of blockchain has introduced the smart contract. Any party in a business can move their contractual obligations to the blockchain network by writing their agreements as the form of smart contract and deploy it to the network. A smart contract is a piece of code (written in special high-level languages for writing contracts in blockchain such as Solidity) which is used to govern transactions exchanges between network nodes. It has been becoming the substitution to traditionally written contracts because of its key elements: autonomy, automation, and decentralization\cite{8}.  

The monetization of the huge amount of available IoT data is a challenging task with respect to automation and scalability. 
Many marketplaces exist that are designed to deal with IoT data using  either centralized or  decentralized architectures, for instance Microsoft Azure, BDEX (\url{bdex.com}), 
and Big IoT Marketplace (\url{http://big-iot.eu/}), a European project to enable IoT Ecosystems where IoT data producers can sell their data.
These are all examples of centralised solutions where a central authority controls and manages the trades between data provider and data buyer.
 
A number of blockchain networks have been used to support IoT data exchange. Some, like Hyperledger \url{hyperledger.org}),Quorum (\url{jpmorgan.com/global/Quorum}) and Corda (\url{marketplace.r3.com}) are private or permissioned.
Hyperledger shows low latency requirements for consensus but does not fully satisfy decentralization goals, while both J.P.Morgan's Quorum  and Corda target the financial sector using different approach, whereby  IoT data are stored off chain and the consensus function is designed to ensure agreements among trade participants. 
 The Ethereum blockchain~\cite{Buterin2014}, used as a testbed for this work,  provides a public platform and automated agreements among interacting parties in the form of smart contracts and supports the development of DApps, making it one of the blockchin-based platforms of choice.

Some decentralized IoT marketplaces also exist. IDMoB~\cite{IDMoB} is designed to trade non real-time and not critical IoT data between IoT data producers and  consumers.
It runs on Ethereum and uses Smart Contracts to manage and control the market and to interact with the Raiden micropayment network.
% The approach is based on uploading an IoT dataset into Swarm and then get a “file handle which is cryptographic hash of the data. The file handle is unique identifier and address of data.”. A data vendor publishes what offers he has on the blockchain and then a data consumer can request these data after get paid for it in off-chain payment channel. Once the payment is done, a vendor sends the key of the data on Swarm and a consumer can access it. A voting system has been used by a consumer in order to vote for vendor as a straightforward feedback mechanism. 

Suliman A. et al \cite{19} propose a marketplace to monetize IoT data using smart contract in the blockchain. Similar to our model, their approach involves sending IoT data through MQTT broker and using smart contracts to manage and settle payments. The main difference with our approach is that a deposit is required before subscription to a topic may take place. This conflicts with our no-trust assumption, as leaving a deposit ahead of receiving goods is likely to be viewed as risky by the buyer.

% First, an IoT device owner has to interact with the smart contract to create a contract with the broker including the topic details where the IoT data sending through. A consumer on the other hand make a deposit to this contract to subscribe from this topic. Then, an IoT device owner send token and the duration of accessing data. An off chain connection will be created between the consumer and the MQTT broker to send the data off chain. Once the duration over, the consumer disconnects or thain e deposit consumed, the connection is stopped. The approach does not give the consumer the flexibility to set the duration he needs to consume the data while it depends on how much he deposited. The subscription still until either the consumer disconnects or the deposit is used up. If the consumer decides to consume more data and no deposit has, he has to deposit again which cost some gas and considered as extra cost. it could be avoided if the consumer has a clear agreement with the desired period before the data exchange.

Huang Z. et al.'s decentralized platform for IoT data exchange~\cite{12} comes close to addressing issues of mis-trust amongst participants, and similar to our approach, data is exchanged off-chain and made available to buyers once the contract is in place. However, the data to be purchased is stored, making this solution unsuitable for streaming. Furthermore, no guarantees are offered to ensure that the data is genuine, so advance payment i.e. to get access to data download is risky.

% presented a decentralized platform for IoT data exchange using blockchain technology in order to achieve the trustworthy and the transparency. They proposed a platform for data provider and data buyer to exchange data without the need for the mutual trust based on blockchain technology advantages. The intrinsic data exchange is done out of the blockchain network while the only data exchange details are recorded and kept in the blockchain. A data provider uploads the data off chain, the system’s smart contract will create a data object and id referred as “DAT”. When a consumer request for accessing data, a data producer will provide some conditions to the demander which has to be met in order to get access. Once the consumer has met theses conditions, a consumer will be put in the list of authorized access od the data with DAT id and con download the data either from URL, FTP or PEP. This marketplace is not designed for a real-time data and it cannot be classified as a marketplace for IoT data streams while if the period of accessing the data is unlimited once the consumer has the link to download the data. In addition, there is no guarantee that the link given to the consumer has the correct data that consumer needs, so it could be a garbage or damaged data.   
	
% Blockchain technology can leverage fair and trust marketplace for IoT data to be exchanged, but it is not adequate to manage and control the data exchange out of the blockchain network. While there is no mutual trust between data producer and data consumer before involving in a trade, so there are some risks for both of them to involve in a trade with party not trusted. it is necessary to guarantee that both of them behave honestly in off-chain data exchange and assess all their trades involved to help to take a decision for future trades.   

Finally, a recently proposed alternative blockchain provider, IOTA (\url{iota.org}), announced their support for decentralized marketplaces  at the end of 2017, with the goal of ``enabling a truly decentralized data marketplace to open up the data silos that currently keep data limited to the control of a few entities''. 
One distinguishing feature of this solution is that, unlike others cited above, here the IoT data is actually stored in the blockchain (or IOTA's version of it, called the Tangle~\cite{Popov2016}. 
To the best of our knowledge this solution has not yet been released.

% The project is essentially a blockchain that was created specially for the IoT and use their own unique public ledger architecture called Tangle (directed acyclic graph data structure to store transactions). 
% It is designed to cope the blockchain limitation in scalability where in IOTA, all real-time IoT data are stored on the chain. 
% unlike the IOTA project \cite{20}, there is no need to record all IoT data in the chain and only need to store the metadata and the trade details between a producer and a consumer as a trade agreement.

Trust and reputation management is not directly addressed in this paper, however a trust management model should also be established as part of the marketplace. Existing trust frameworks can be used on top of our infrastructure.
Yan et al. \cite{Yan2014a}, for instance, explore the notion of trust across the IoT platform layers (physical sensing, network, and application layers), with the focus on a wide range of properties from security to goodness, strength, reliability, availability, ability of data. However, their survey largely overlooks issues of trust amongst participants in a data marketplace, i.e., in the context of data exchange transactions.
%
%\note{ Roman, D., \& Stefano, G. (2016). Towards a Reference Architecture for Trusted Data Marketplaces: The Credit Scoring Perspective. In 2016 2nd International Conference on Open and Big Data (OBD) (pp. 95–101). IEEE. https://doi.org/10.1109/OBD.2016.21}
More directly useful in our setting, is Roman and Gatti's study of trust in data marketplaces \cite{7573695}, based on \textit{credit scoring}, where a direct connection is made to the use of blockchain technology with data trading.

\section{Marketplace Model} \label{sec:MKModel}

\subsection{Brokered IoT Data Exchange} \label{sec:daga-exchange}

\begin{figure}
  \caption{Centralized Brokered IoT Data Marketplace Architecture}
  \label{fig:brokered-data-exchange}
  \includegraphics[scale=0.4]{Cent}
%   \centering

\end{figure}

We assume, following standard IoT data streaming practices, that the exchange of streaming data between any pair of  participants, i.e., a data Provider  $P$ and a Consumer $C$, is  mediated by some transaction-agnostic broker infrastructure, such as the one shown in Fig.~\ref{fig:brokered-data-exchange}.
In this data transfer model, the stream is broken down into discrete message batches. Providers tag their messages with \textit{topics} that uniquely identify that Provider's stream. 
A Consumer is allowed subscribe to a topic subject to the conditions set in a Trade Agreement, as described below.

In our previous work~\cite{Missier2017} we assumed initially a network architecture where the broker is a trusted component that can be relied upon to generate truthful data exchange reports, which in turn can be used to settle disputes between producers and consumers  (the ``cubes'' in Fig.~\ref{fig:brokered-data-exchange}).
In this scenario, the Smart Contract is simply in charge of settlement given the reports. 
In the same paper, we then proposed a more ambitious trust-less model where the task of generating reports is left to each participant. In this case the Smart Contract has a difficult task because the report themselves cannot be trusted, and disputes cannot be settled by ascribing certain responsibility to either participant.


% % Recent days, sharing the economy with IoT data has increased the demand for establishing IoT data marketplace. In brokered IoT data marketplace scenario as shown in Figure\ref{Cent} , data providers have to push their data through a centralized broker which then data consumers subscribe through predefined topics. A broker has the ability to count the number of messages published by a producer and subscribed by a consumer for all producers and consumers participated in the marketplace, which is dependable as a data source for the settlement process. The settlement process relies on the data metering from the broker and applies the settlement straightforward for both producer and consumer. 
	
% In the above centralized reward data marketplace model, with the assumption of removing the trust from the broker, it could be a threat to alter or change some of data traffic metering between producers and consumers which would affect on the data authenticity, therefore the settlement process would be unsatisfied. This could be a vital factor in the marketplace sustainability especially with the absence of the trust and the lack of other secondary dependencies could be used to check the missed or altered metering data of the trade.
% The centralized infrastructure, as well as the third party- intermediators for data exchange, is not a centralized trusted component. It has been watched by the researchers that the trust issue has limited sharing data and killed the enthusiasm to be part of unreliable marketplace. While some of the current platforms have strength their model by many security mechanisms such as access control and authentication privacy, they still depend on the third party \cite{12}.

\subsection{Model Elements}

In this work we work around these difficulties, as we do not require the broker or the participants to generate any report at all. 
Instead, the broker is simply a network element. The goal of the marketplace is twofold. 
Firstly, to enable trading of streaming data through the broker while offering guarantees, i.e., regarding the max  loss incurred by either of them in case of adversarial behaviour. And secondly, to resolve disputes about the amount of data exchanged.
%
To achieve this, we augment the data exchange with the exchange of \textit{data receipts} between $C$ and $P$, which occurs at regular intervals and throughout the duration of the data stream. Such receipts are exchanged as part of transactions that are mediated by a smart contract, denoted \smartc, on the blockchain. 
The length of the exchange interval, denoted as Batch Size or \bs, is set at the time of trading agreement negotiation. 
As we will see, this parameter enables  $P$ to control the level of risk they are prepared to tolerate given limited trust in $C$.

The model consists of the following elements:
\begin{enumerate}
	
	\item The description of data offered by a Producer;
	
	\item A trade agreement, which includes details of the data to be exchanged and the exchange protocol, the corresponding market value, and additional parameters such as the \bs{} mentioned above;
	
	\item A protocol for the exchange of data receipts, which includes both parties in addition to a neutral smart contract;
	
	\item A reputation model, which allows a score to be assigned to every pair $P$ and $C$ of participants at the end of each transaction they are involved in. 
	Participants may use reputation scores to asses the risk of entering into an agreement with an untrusted participant.

\end{enumerate}

In this paper we are concerned primarily with (1-3), which are described in detail below. Regarding (4), we are going to assume that a reputation model is in place and that a up-to-date score is associated with each participant, without concern for how it works. The design of a customised reputation model is the object of our ongoing work, and it is beyond the scope of this paper.

As we will see, the smart contract is responsible for each transaction associated with (1-3), and specifically for recording (i) the specification of the data offering, (ii) the trade agreement, and (iii) each data receipt.

\subsection{Data Offering}

The first function of the smart contract is to let data Providers publish their data offerings on the blockchain, where they can be then discovered by prospective Consumers. 
As mentioned earlier, a data stream consists of a sequence of messages uniquely identified by  a provider's topic, and a data offering describes the type of stream and specifies how to subscribe to the stream.
Specifically, a data offering $\doff = \langle \topic, \ti, \mr, \up \rangle$ includes, in addition to the topic \topic, a specification of (i) the time interval \ti{} during which the offer is valid, (ii) the expected streaming message rate \mr, eg. in messages/time, (iii) the unit cost \up{} of each message in the stream.

%
%
%
%In the marketplace, a data provider to what data he has to let consumers choose from and subscribe. A producer should interact with the marketplace smart contract to deploy his offers in the network. Each offer is a new offer object stored in the blokchain. Each offer consist of: (1) The time interval of this offer ($TI$). It should set the start and end time that the he is willing to provide this data. (2) A message rate (Rate)(in a time unit e.g. seconds) that would be used to send the data through the broker (3) The price of a message unit (UP), and (4) the minimum reputation ($Min_{rep}$) for candidate consumers who can choose this offer (optional).  
%

%\subsection{Marketplace Stakeholders}
%
%The decentralized marketplace architecture introduces a new layer that uses the blockchain technology upon the brokered IoT data protocol, shown in Figure 1. It is a public platform for any owner has a real-time IoT data stream and willing to monetize his data for economic benefits. Also, it is for anyone who has the desire to have a trade with data stream provider for various purposes such as analysis, aggregation, ...etc. 


\subsection{Trade Agreement} \label{sec:agreement}

The trade agreement is a legally binding contract (we use the term ``agreement'' to avoid confusion with smart contracts) between a producer and a consumer, which defines the terms of the data exchange.
An agreement comes into force when (i) it is signed by both parties using their blockchain account keys (Ethereum in our implementation), and (ii) a smart contract transaction containing the agreement is committed to the blockchain, at which point it can no longer be amended.
The agreement contains (i) a specific data offering \doff{} and (ii) a time interval $\tati$, contained within the time interval \ti, during which the agreement is in force.
For instance, $C$ may want to subscribe to a portion of an event that is offered over a long period of time.
We denote the total price as $\tp = \up \cdot \tati$ and the total number of messages in the agreement as $\etm = \mr \cdot \tati$.

%
%
%
%Each trade agreement TA consists of: a producer address in the blockchain, consumer address, a message rate (Rate), a data batch size (BS), a trade time interval with specifying the start and the end time (TATI), the total messages which are expected to be delivered in the time interval (ETM) and the total data price which is credit from consumer balance (TP). 

\subsection{Data Receipt protocol}  \label{sec:protocol}

Once the trade agreement is in force,  $C$ is allowed to subscribe to $P$'s stream.
Under normal circumstances and when both parties comply with the agreement and data transfer takes place as expected, at the end of the \tati{} interval $C$ informs \smartc{} that the agreement has been fulfilled, and \smartc{} proceeds to settle the payment as per the agreement.
Suppose however that $C$ fails to inform \smartc. This may happen because $C$ actually failed to receive some of the data in the stream, or because it fraudolently \textit{claims} not to have received the data.
In our model we assume that \smartc{} is unable to distinguish between these two events, because there is no requirement for the data broker to keep a (verifiably truthful) log of its message delivery.
In this situation, the only possible course of action for \smartc{} is to believe $C$'s claim, and to withhold $P$'s payment as a consequence.
Thus, assuming minimal accountability on the broker and no trust amongst participants, $P$ may become the victim of $C$'s fraud.

Our approach to mitigate this circumstance is to introduce \textit{checkpoints} throughout the duration of data delivery. 
The number of messages between two checkpoints is the batch size \bs, which $P$ can configure as part of the agreement negotiation with $C$.
At each checkpoint, $C$ is expected to send a \textit{data receipt} to \smartc{} as part of a blockchain transaction, which acknowledges receipt of one batch of data from $P$.
When the transaction is confirmed, \smartc{} records the receipt and then informs $P$.
Meanwhile, at the end of each batch $P$ will have suspended its streaming to $C$ until it receives the acknowledgment from \smartc.
If $P$ does not receive a message within a certain time limit, it times out and terminates the trade agreement (in practice, $C$'s subscription to the stream is cancelled).
Thus, the data exchange protocol and data receipt protocols are interconnected as shown in Fig.~\ref{fig:batching}.

The timeout is a configurable parameter that reflects the expected time required for a receipt transaction to be confirmed on the blockchain. 
In our experiments we model this time as a random variable, denoted \rtEst, 


%The max loss of data from $P$ depends by the time required for each receipt smart contract transaction to be confirmed on the blockchain.
%When using the simple protocol just defined, $C$, $P$ and \smartc{} interact throughout the data exchange period \ti{} as illustrated i Fig.~\ref{fig:batching}.
%As we can see in the figure, when no fraud occurs $C$ issues a receipt transaction to \smartc{} every \bs{} messages, confirming the number of messages received with the last batch. 
%The time required for this transaction to be confirmed on the blockchain is a random variable whose distribution depends on the network load and the unit gas price chosen to enable the smart contract (in the case of Ethereum). 
%Let \rtEst{} denote the expected value for this distribution, that is, the expected transaction confirmation time.

\subsection{Cost of agreements}  \label{sec:cost}

\mynote{FIXME}

$P$'s objective is to balance the risk of losing data, against the cost In this setting, \textit{data loss} refers 

It is straightforward to see that $P$'s max data loss is simply one batch  of messages.

Thus,  $P$ controls the the max loss of data and thus its risk by setting the value of \bs.
A small value ensures that data loss will be limited to a few data batches (see below for details), while a large \bs{} exposes $P$ to a higher risk of fraud.

\mynote{need to explain provider timeout. rewrite the following sentence: \\
Since $P$ cannot assume that the receipt has been issued until it is confirmed, and data streaming continues during this time, $P$ uses \rtEst{} as a timeout to prevent unbounded losses.
It starts its timer at the end of each batch, and if no message is received from \smartc{} after \rtEst{} has elapsed, it terminates the contract (the broker terminates $C$'s subscription).
}

\mynote{must say something about how reputation scoire is updated at the end of each transaction. just say there is a mechanism but it is out of scope.}



Setting \bs{} correctly for each transaction is critical to achieving a sustainable marketplace, because checkpoints are smart contract transactions and as such, in blockchain models like Ethereum, they each incur a fee. There is therefore a trade-off between the risk of losing data and the cost of engaging in a long-running transaction with many checkpoints along the way.
To make this observation precise, let us calculate the total cost of purchasing a data stream from a provider, as a function of the tunable parameter \bs.

Firstly, in order to participate in the marketplace each participant, in either a consumer or producer role, must register itself with the network. 
This incurs a one-off \textit{registration cost} to execute the smart contract user registration function.

Secondly, the deployment of a trade offer to the network is also implemented as a smart contract function, which again incurs a fee. This is a provider-only cost.

Thirdly, a smart contract fee is paid when a new trade agreement is recorded on the blockchain. This cost is split between $P$ and $C$.
%

While these are all one-off costs and can be regarded as constant, the receipt protocol generates one transaction for each batch, for a total of 
\[\bn = \ceil[\bigg] {\frac{\etm}{\bs}} \]
transactions (the ceiling accounts for fractional batches at the end of the \tati{} period). 
Assuming the Ethereum cost model with gas unit price \gup{} and gas consumption per transaction \gt, the total cost \rc{} due to the receipt transactions is 
\[\rc = \bn \cdot  \gt \cdot \gup    \] \label{RC}

\mynote{Shaimaa:  the equation $\bn=\frac{\tati}{\bt+\rt}$ assumes that \tati{} includes all the \rt{} intervals, but I think in our latest discussion we have decided this is not the case -- think of \tati{} as the duration of a game, that should not include the ``advert'' time \rt{}.
}

% instead \bn{} is defined simply as above, which means that if $\bs=C_{rep} \cdot \etm$, then the cost \bn{} reduces simply to $1/C_{rep}$.


This cost is inversely proportional to \bs{}. In our experiments we have assumed $\bs= f(\crep) \cdot \etm$ where   $f(\crep{}) \leq 0.25$ is a function of $C$'s current reputation score, \crep.
In this case, \bn{} simplifies to $\bn =  1/f(\crep{})$.
Function  $f(\crep{})$ is a parameter in the model and can be chosen to either amplify or reduce the effect of reputation. In our experiment we have used a logarithmic function:  $f(\crep{}) = \ln(\crep + 1)/2.77 $. This function has the effect to normalize \bs{} in such a way that 
%and divided by a constant number $2.77$ where
the minimum number of receipts is 4.
%, i.e., four receipts are geneso a consumer may interact TATI-quarterly for a transaction and also reduce the maximum risk a producer could lose to the quarter. 

\mynote{sentence below needs to be  clarified}
Note that the minimal number of receipts could be $\bs = 25\% of ETM $ for $f(1)$ is $4$ which is fairly bands the less reputations into smaller groups of the same number of receipts based on the their reputations.

\mynote{will explain why 2.77 in the implementation}

% use a linear function with \mynote{can you complete this?  I know you have used $\crep \cdot \mr$ but as discussed, this looks incorrect}

\begin{figure*}
	\caption{Data receipt protocol Interactions between $C$, $P$ and \smartc{}}
	\label{fig:batching}
	\includegraphics[width=.7\textwidth]{Dis}
	\centering
\end{figure*}

%\anote{EDITED  ABOVE -- ORIGINAL BELOW -- PM}

%\subsection{Data Batching}
%With the absence of central trust, our approach is a receipt-based model. A consumer is required to send a receipt for every data he received, for this, the real-time stream has to be divided into batches.
%
%A data batch is a group of real-time IoT data which is divided into batches in order to has a checking point after each batch. The checking point checks if the batch has been delivered to a consumer. After each batch has been sent, a consumer has to send a receipt to what he received from the previous batch to the smart contract. An acknowledgment will be sent from the smart contract telling the producer that the data which are sent has been delivered then the producer resume sending the next batch. Once a trade agreement expires, a producer stop streaming the data and a consumer has to send a last receipt to settle the trade.
%
%A receipt is a receipt object sent by a consumer to the smart contract. It reports the number of messages delivered by a consumer for batch received.


% While every receipt sent by a consumer needs some gas in order to interact with the smart contract, the less receipts sent means the less gas consumed in smart contract invocation. It is worth to note that specifying the size of the batch is auto-calculated based on a consumer reputation in the marketplace. In other words, a consumer who has high reputation, has large amount of data in a batch which leads to less checking point and therefore less smart contract invocation. 

%
%\mycomment{The batch size is a linear function of a consumer reputation in the marketplace multiplied by a message rate $Rate$} {there is a problem here BS cannot be a rate}.
%	
%	\mycomment{It has to mention that in case of dishonest consumer, a producer will lose at maximum one batch size,}{this is true only if the producer can suspend data delivery and then continue. But i think this is not a viable model.}
%%
%\begin{lstlisting}[basicstyle=\smaller, frame=single]
%ETM = TATI * Rate      
%
%# ETM: Estimated Total Messages in a Trade 
%# TATI: TA Time Interval
% 
%if C_rep =0 Then:
%    A producer has to set the BS manually.
%else:
%    BS = Rate * C_rep
%
%\end{lstlisting}

%It is worth to note that the data stream is a time sensitive which means it will lose its value once it is divided into batches. The time taken between two batches (RT) as shown in figure \ref{Stream} could be a waste time for a consumer to wait for the receipt to be processed. Because of the batch size function is a linear, a consumer with a high reputation score would use most of the contract time to receive data. An honest consumer has batch size larger than dishonest consumer which means less time consumed in receipts processing and therefore more real-time data received. It is a direct proportional between consumer reputation and using the value of the real-time data stream. 

%  While producers generate a real-time data stream which has to be delivered continuously and on the time of its generation or otherwise it will lose its value, some data is tolerant to be interrupted. Because of the needs of the checking point in our model to process batches receipts sent by a consumer, both streams type have to be divided into batches and checking point for receipts will be after each batch. 



% \begin{figure}
%   \caption{Continuous Data Stream}
%   \label{Con}
%   \includegraphics[scale=0.2]{Con}
%   \centering
% \end{figure}

% \begin{figure}
%   \caption{Discontinuous Data Stream (Interrupted tolerance)}
%   \label{Dis}
%   \includegraphics[scale=0.3]{Dis}
%   \centering
% \end{figure}


% % Start the sub figure
% \begin{figure}%
%     \centering
%     \subfloat[Continuous Data Stream]{{\includegraphics[width=9cm]{Con} }}%
%     \qquad
%     \subfloat[Discontinuous Data Stream (Interrupted tolerance)]{{\includegraphics[width=9cm]{Dis} }}%
%     \caption{Data Stream Categories}%
%     \label{ConDisStreams}%
% \end{figure}
%%%%%%%%%%%%%%%%%%%%%%%%%%%%%%%%%%%%%%%%%%%%%%%%%%%%%%

% \anote{I think this is all we need for this section. much simplified because behind the many equation the storyr is actually very simple, so I commented out the original text--- but please look at the comments.}

%%%%%%%%%%%%%%%%%%%%%%%%%%%%%%%%%%%%%%%%%%%%%%%%%%%%%%%



%\subsection{Last Data Package}
%As agreed in the TA, the data stream will be stopped once the subscription expires. Therefore, if there was a batch has been starting and has not been completed yet, the data provider has to stop streaming the data and the consumer must send a receipt for data portion he received, lets call it (Last Data Package in the trade LP) as shown in figure \ref{Stream}. A processing time for the receipt (LPRT) will be out of TATI which means the smart contract will wait LPRT time to receive the last receipt and then  add it to the total receipts have to make the settlement. It is expected by the smart contract to receive a receipt for the LP within a period of time $LPRT $where $LPRT = RT$.

%\subsection{Participants' loss} \label{ParLoss}
%In the marketplace protocol, and based on the assumption that the data providers are honest while they have no incentives not to sell their data in the marketplace, they have to take a risk when decide to involve in new trades in this marketplace. A producer could reduce his risk and loss by set a minimum reputation for candidate consumers when he pushes an offer in the blockchain, and also can updates the values anytime based on the current market state. 
%
%\mynote{It should mention that a consumer might cheat and does not send a receipt to what he received, and therefore the data owner (producer) will loses the unreported data and the consumer will be rated with low reputation score based on the proposed reputation model of this marketplace.}
%
%A consumer should send a receipt within RT while the producer waiting for the acknowledgment to resume sending the next data batch. If the consumer does not send the receipt or does not report the correct amount of data delivered, the trades stops immediately and a producer can only lose the price of at maximum one full batch as the following where Max and $ P_{Loss} $ are Maximum Data Loss and Producer Loss, respectively:
%\vspace{-0.1 cm}
%\mathleft
%\begin{align}
%\textbf{a) } & Max = BS \nonumber\\
%\textbf{b) }  & 0 <= P_{Loss} <= ( Max \times UP )\nonumber
%% \textbf{c) }  & Max_CLoss = Max_DLoss \times Price \times Price 
%% \nonumber\\
%% \textbf{d) }  &  >= P_CLoss <= Max_CLoss \nonumber
%\end{align}

%\subsection{Settlement Process}
%
%The settlement process is the task of fulfillment the financial contractual obligations of a trade.
%At the end of each trade, the price of the total data delivered (AP) should be taken out from the consumer deposit and transferred to the producer account,  and the rest (if any) will be returned back to a consumer account. When a consumer interact with the smart contract to create a new TA, he deposits tokens for the expected total data price (TP) could to be delivered during the TATI. The expected total data  ETM  and AP are calculated as the following:
%
%\vspace{-0.2 cm}
%\mathleft
%\begin{equation}
%\textbf {$ETM = TATI \times Rate. $}    
%\end{equation}
% 
%\vspace{-0.5 cm}
%% OR \mynote{$ETM = TATI//BS \times Rate$} This exclude the LP ( does not work )
%
%\mathleft
%\begin{equation}
%\textbf {$TP = ETM \times UP $} 
%\end{equation}
%
%\vspace{-0.5 cm}
%
%\mathleft
%\begin{equation}
%\textbf {$AP = UP \times \big(BS \times BN + LP\big)$} 
%\end{equation}
%

%where UP, BS, BN and LP is the price per message unit, the batch size, the batch number and last package data, respectively and $ AP \subset TP$.
%
%Each trade parties in the marketplace costs some tokens for trading. Firstly, everyone in the marketplace has to register in the network by invoking registration method in the smart contract. This cost will be deducted once from their balances and it is called a \textit{registration cost}. \textit{ An offering cost} is the cost deducted from a producer account in order to deploy his offer to the network. He invokes the offer method from the smart contract with his offer details as a transaction payload. \textit{ A set up cost} is deducted for each trade between a producer and a consumer which include all expenses of making an order and creating a new TA from consumer side, and do the data batching and approve the created TA from the producer side. \textit{An Authentication cost} is the cost paid for sending a receipt for every data batch received by a consumer. This cost will be deducted only from consumer balance for sending receipts. This cost will be paid  as receipts cost as the same as the number of batches delivered plus the last package (if any) by a consumer in a trade. Producer and consumer balances will be as follows for trade $T_{i}$ and $offer_{i}$:
%
%\vspace{-0.3 cm}
%
%\mathleft
%\begin{equation}
%\textbf{$P_{balance} -= ( Offering Cost_{i} + Setup Cost_{i} )$} 
%\end{equation}
%
%\vspace{-0.5 cm}
%
%\mathleft
%\begin{equation}
% \textbf{$C_{balance} -= ( C_{cost_{i}} + AP{i} )$}   
%\end{equation}
%
%\vspace{-0.5 cm}
%
%\mathleft
%\begin{equation}
%\textbf{$ C_{cost_{i}} = Setup Cost_{i} + Authentication Cost_{i} $} \label{C_costEq}     
%\end{equation}
%
%
%
%Note: $ Registration Cost$ is excluded because we assumed it was taken once from their balances in their first interaction in the marketplace. 
%
%We will use the following as abbreviation; Gas Used for Registration as GUR, Gas Used for Setup as GUS, Gas Used for Authentication for sending the receipt as GUA, Gas Price by Gwei as GP, Receipt number in a trade as RN, batch number as BN, Last data package LP, Batch sending time as BT, receipt processing time as RT, and time between two batches as B2B.
%
%\vspace{-0.5 cm}
%
%\mathleft
%\begin{equation}
%\textbf{$ Registration Cost = GUR \times GP$ }   
%\end{equation}
%
%\vspace{-0.5 cm}
%
%\mathleft
%\begin{equation}
%\textbf{$ Setup Cost_{i} = GUS_{i} \times GP_{i} $}    
%\end{equation}
%
%\vspace{-0.5 cm}
%
%\mathleft
%\begin{equation}
%\textbf{$ Authentication Cost_{i} =  RN_{i} \times GUA_{i} \times GP_{i} $}    
%\end{equation}
%
%\vspace{-0.5 cm}
%
%\mathleft
%\begin{equation}
%\textbf{$ RN _{i} = BN _{i} + LPN _{i} $}   
%\end{equation}
%
%\vspace{-0.5 cm}
%
%\mathleft
%\begin{equation}
%\textbf{$ LP _{i} = TATI _{i}  \%  B2B _{i} $}   
%\end{equation}
%
%\vspace{-0.5 cm}
%
%\mathleft
%\begin{equation}
%BN _{i} = \frac{TATI _{i}}{B2B _{i}}
%\end{equation}
%
%\vspace{-0.5 cm}
%
%\mathleft
%\begin{equation}
%\textbf{$ B2B _{i} = BT _{i} + RT _{i}$}    
%\end{equation}
%
%\vspace{-0.5 cm}
%
%\mathleft
%\begin{equation}
%BT _{i} = \frac{BS _{i}}{Rate _{i} } 
%\end{equation}
%
%\vspace{-0.5 cm}
%
%\mathleft
%\begin{equation}
%\textbf{$ BS _{i} = C_{Rep} \times Rate $}    
%\end{equation}
%
%\vspace{0.5 cm}
%
%From all the above equations, we could rewrite formula (\ref{C_costEq}) as follows:
%
%% \mathcenter 
%\begin{equation} \label{FCostEq}
%\boldmath C_{cost} = \bigg\{ \big[ \frac{TATI}{C_{Rep}+RT} + LP\big] \times GUA + GUS \bigg\}\times GP
%\end{equation}

%
%Where $C_{cost}$ is a consumer cost in a trade.

\section{System Architecture and Workflow} \label{Arch}
%%%%%%%%%%%%%%%%%%%%%%%%%%%%%%%%%%%%%%%%%%%%%%%%%

\begin{figure*}
  \caption{System Sequence Diagram}
  \label{SSD}
%   \includegraphics[scale=0.31]{SystemArchWithBackground}
  \includegraphics[width=.7\textwidth]{SystemArchWithBackground}
%   \includegraphics[width=300pt]{SystemArchWithBackground}
  \centering
\end{figure*}

The system is divided into two layers; Data Layer and Blockchain Layer. In data layer, all IoT data will be transferred off chain from data producers to data consumers through brokers in-between, while in the blockchain layer, the negotiation and the the final trade agreement will be created and recorded. The blockchain layer is represented as a set of automated smart contracts written in Solidity (high-level language for writing a smart contract on Ethereum network and executed in Ethereum Virtual Machine EVM in Ethereum network).

As shown in Figure\ref{SSD}, the marketplace starts that each participant must register in the blockchain by interacting with the smart contract and calling the register method. A data provider publishes/updates his offers to what data he has. All offers will be stored in the blockchain and it is visible for anyone. On the other hand, a data consumer checks list of available offers , and then when he decides to subscribe data, he has to interact with the smart contract with his offer’s choice and the time interval who would like the subscription start and end. A new TA is created and signed by consumer by his private key in the blockchain and has been deployed. Next, a data provider checks this consumer’s reputation in the marketplace and then set the data $BS$ if it is zero or it would be calculated automatically, add it to the TA and then deployed it to the blockchain with his signature. Now, we have a new TA signed by both trade’s parties and visible to anyone and can not be amended. 

The data provider at the start time of the subscription starts sending the data batches as agreed in the TA. When a first batch has been sent, a consumer should send a receipt to the smart contract reporting what he received. A smart contract will check if the data received is equal to $BS$ as stated in the agreement. Meanwhile, the provider is paused and waiting for the acknowledgment to be triggered. 

It has to mention that while a $P$ can not resume sending the next batch until receipt transaction confirmed by miners, $SC$ would set the the timeout \rtEst{} before terminating the trade's contract and trigger the acknowledgment that a $P$ must stop sending batches.In this stage, the settlement function release the tokens to the producer as mush as reported in the receipt and return back the rest to the consumer balance. Eventually, a reputation score is updated and eventually the trade is labeled failed and a a consumer score will be affected based on the last missed receipt. The reputation score has been updated by the end of each trade in such a way recalculate the new score to be used in the marketplace for next trades, but its mechanism is out of this paper scope.

% %%%%%%%%%%%%%%%%%%%%%%%%%%%%%%%%%%%%%%%%%%%%%%%%%

\section{Implementation and evaluation} \label{sec:evaluation}

The decentralized brokered IoT data marketplace model has implemented the trade management and the settlement service as set of smart contracts in the blockchain written by scripting, high-level language \textit{Solidity} in Ethereum platform. 

For experimentation and evaluation purposes, we have written the smart contracts and then deployed it in a private Ethereum test network. We used Ethereum web browser based IDE Remix (\url{remix.ethereum.org}) to write, deploy and connect to the private chain through Remote Producer Call (RPC) protocol. We have used the fake accounts with their balances provided by Remix as a trades’ participants. 

Each account has to interact with the smart contract to register in the network by providing an id and a role in the marketplace, i.e. producer or consumer. Then, a producer incur gas to deploy offer(s), and on the other hand a consumer interacts to get an offer. Another gas consumption for the consumer is in interacting with the smart contract to request trade based on provided offers, initiate a new TA and deposit the full cost of the trade based on the estimated messages are expecting to be delivered ETM. In this stage, BS has been set based on the consumer reputation and eventually sign the current TA with this candidate consumer. 

Based on the receipt protocol proposed, a consumer has to send a receipt $BN$ times for every batch received.

Table \ref{GasTable} shows the cost categories whose either producer or consumer or both incur in the marketplace. We have measured the gas consumption using Remix debugger which provide a consumed gas for every transaction done. Also, it can be calculated by monitoring the balances of participants and check the differences before and after invoking the smart contract method. 

\begin{table}[h]
\caption{shows transactions cost in each cost category (in Gas)}
\label{GasTable}
\begin{center}
% \begin{adjustbox}{max height=4 \textheight, width=0.55\textwidth }
\begin{tabular}{|m{15mm}||m{20mm}||m{15mm}||m{15mm}|}
\hline
\textbf{\makecell{Cost\\Category}}&
\centering\textbf{Operation}&
\textbf{\makecell{Producer\\Gas\\Consumption}} & 
\textbf{\makecell{Consumer\\Gas\\Consumption}} \\
\hline
\centering\makecell{Registration \\Cost} & \makecell{- Register in\\ the network} & 204739 gas & 199093 gas 
  \\
\hline
\makecell{Offering \\Cost} & - Deploy an offer & 491862 gas & - \\
\hline
\multirow{2}{*}\centering{\makecell{Setup\\Cost}} & \makecell{- Make an order \\and create \\ a new TA} & - & 620865 gas \\

& \makecell{- Set Batch size \\and sign off \\ the TA} & 82063 gas & - \\ 
\hline
\makecell{Receipt\\Cost} & - Send a receipt & - &  144367 gas \\

\hline
\end{tabular}
% \end{adjustbox}
\end{center}
\end{table}
% 

The settlement is done by the settlement smart contract when the trade ends. Based on the protocol proposed, $BS$ is $f(C_{rep})$ which is directly proportion to $C_{rep}$. We have done many experiment to pick the constant number $f(C_{rep})$ and set the minimum receipts could have. For the purpose of evaluating, we have fixed the parameters TATI with 1 day (24 hours), $MR = 100 m/s $ and use three different constants $1.38, 2.10, 2.77$ to to normalize $BS$ in such way that the minimum number of receipts are $2,3 and 4$, respectively. Table\ref{ReceiptNumber}.


\begin{table}[h]
\caption{Minimum number of receipts with three different constant values in $f(C_{rep})$}
\label{ReceiptNumber}
\begin{adjustbox}{width=0.48\textwidth}
\begin{tabular}{|c||c||c||c|}
\hline
\textbf{Reputation} &\textbf{$\ln(\crep + 1)/1.38$} & \textbf{$\ln(\crep + 1)/2.10$} & \textbf{$\ln(\crep + 1)/2.77$} \\

\hline
0.1  & 15 & 21 & 29 \\

\hline
0.2  & 8 & 11 & 16 \\

\hline
0.3  & 6 & 7 & 11 \\

\hline
0.4  & 5 & 6 & 9 \\

\hline
0.5  & 4 & 5 & 7 \\

\hline
0.6  & 3 & 4 & 6 \\

\hline
0.7  & 3 & 3 & 6 \\

\hline
0.8  & 3 & 3 & 5 \\

\hline
0.9  & 3 & 3 & 5 \\

\hline
1  & \textbf{2} & \textbf{3} & \textbf{4}  \\

\hline
\end{tabular}
\end{adjustbox}
\end{table}


 The constant $1.38$ in table\ref{ReceiptNumber} shows that the minimum number of receipts is $2$ for $C_{rep} = 1$ and increases to $3$ for $0.5 \leq C_{rep} \leq 0.9$  which is not fair enough to have the same number of receipts. Wile in the constant $2.10$, the minimum receipt numbers has became for reputation band  $C_{rep} \geq 0.7$. The constant $2.77$ shows that the minimum number of receipts for $C_{rep} = 1$ while it bands the reputations fairly enough for $0.6 \leq C_{rep} \leq 0.7$ with $6$ receipts and for $0.8 \leq C_{rep} \leq 0.9$ with $5$ receipts.
 
Because the focus of our experiment is the relation between the cost and the consumer honesty, We have chosen the constant $2.77$ which shows fair distinguish in consumer bands based on their reputation. On the other hand, it reduces the loos a producer could have in case a receipt has not been sent. 

The increase in a number of batches received means that a consumer will incur more gas. If we assume that a producer has no incentive not to send the data as agreed in the TA, and with a guarantee the quality of service, a trade would be failed if a consumer was not send a receipt or was not honest in reporting the exact number of messages he received. It is crucial for a marketplace sustainability to have an honest consumer in order to have successful trades, and also for the consumer financial affairs to be avoided from many smart contract invocations and therefore extra cost (receipt cost).

Because $GUP$ determines the  time of the transactions in the blockchain to be confirmed by miners, a consumer has the potential to increase the $GUP$ in order to process his transaction faster and therefore he will have more time to receive more batches. 

It has to notice that the transaction of sending receipts to the smart contract has to be processed in the period of time, so a consumer may increase $GUP$ of his transactions to give it the priority to be picked by miners. Miners who will validate the transactions usually follow the strategy of picking the transactions’ heights $GUP$ to be included in the next block in order for high rewards. Because of the transactions fees goes to the miners as rewards for their verification, a consumer may set $GUP$ high enough to encourage miners to select his transaction to be involved in their next block. The increase in $GUP$ may contribute in decreasing $RT$ and therefore more time to receive more data messages within $TATI$

A minimum and the maximum $GUP$ in the network could be known and limit in ETH Gas Station (\url{ethgasstation.info}).  This is a tool to understand the conditions of the current gas market and current policies of network miners. 
Based on the current condition of the network at the moment of writing, the recommended gas prices from Gas Station is shown in table \ref{GasPriceTable}. It shows the maximum time taken by miners to confirm the transaction for each $GUP$. In addition, the Gas Station provides the median time of transaction confirmation for each $GUP$.

\begin{table}[h]
\caption{Gas Prices and Speeds}
\label{GasPriceTable}
\begin{center}
\begin{tabular}{|c||c||c|}
\hline
\textbf{\makecell{Gas Price\\ (Gwei)}} &\textbf{Speed} & \textbf{Median Speed}\\

\hline
1.6  & SafeLow (\textless30m) & 4.3m \\

\hline
4.2  & Standard (\textless5m) & 2.5m \\
\hline
7.2  & Fast (\textless2m) & 0.8m \\

\hline
\end{tabular}
\end{center}
\end{table}

For the purpose of evaluation, we have used the three different $GUP$ for different consumer reputations to calculate the $RC$ (\ref{RC}). 


\begin{table}[h]
% \begin{center}
\caption{Shows the RC in USD and Data Percentage Delivered for consumer reputations for 1 day trade with three GUP values, MR=100 msg/sec. (1 Eth $\approx$ 202.70 USD)}
\label{CostTable}
% \begin{center}
\begin{adjustbox}{width=0.48\textwidth}
\begin{tabular}{|c||c||c||c||c||c||c|}
\hline
\textbf{Consumer} & \multicolumn{2}{c}{\textbf{GUP= 1.6 Gwei}} &
\multicolumn{2}{c}{\textbf{GUP= 4.2 Gwei}}  & 
\multicolumn{2}{c}{\textbf{GUP= 7.2 Gwei}}\\ 



\textbf{Reputation} & \textbf{\makecell{Cost\\in USD}} & \textbf{\makecell{Data\\ Percentage}} & \textbf{\makecell{Cost\\in USD}} & \textbf{\makecell{Data\\ Percentage}}
& \textbf{ \makecell{Cost\\in USD}} & \textbf{\makecell{Data\\ Percentage}}\\ 
\hline
$0.1$ & $ 1.465 \$ $ & $92.236 \%$ &  $ 3.970 \$ $ & $95.313 \% $ &  $ 7.016 \$ $ & $ 98.444 \%$\\

\hline
$0.2$ & $ 0.903 \$ $ & $95.81 \%$ &  $ 2.372 \$ $ & $97.57 \%$ &  $ 4.277 \$ $ & $99.17 \%$ \\

\hline
$0.3$ & $ 0.716 \$ $ & $97.01 \%$ & $ 1.881 \$ $ &$98.26 \%$& $ 3.224 \$ $ & $99.44 \%$ \\

\hline
$0.4$ & $ 0.623 \$ $ & $97.61 \%$& $ 1.635 \$ $ & $98.61 \%$&$ 2.802 \$ $ & $99.55 \%$ \\

\hline
$0.5$ & $ 0.529 \$ $ &$98.20 \%$ & $ 1.389 \$ $ &$98.95 \%$ & $ 2.381 \$ $ & $99.66 \%$ \\

\hline
$0.6$ & $ 0.482 \$ $ &$98.51 \%$ & $ 1.266 \$ $ &$99.13 \%$ & $ 2.170 \$ $ &$99.72 \%$\\

\hline
$0.7$ & $ 0.482 \$ $ &$98.51 \%$ & $ 1.266 \$ $ & $99.13 \%$& $ 2.170 \$ $ & $99.72 \%$\\

\hline
$0.8$ & $ 0.435 \$ $ &$98.81 \%$ & $ 1.143 \$ $ & $99.30 \%$& $ 1.960 \$ $ &$99.78 \%$ \\

\hline
$0.9$ & $ 0.435 \$ $ & $98.81 \%$& $ 1.143 \$ $ & $99.31 \%$& $ 1.960 \$ $ & $99.78 \%$\\

\hline
$1.0$ & $ 0.388 \$ $ &$99.10 \%$ & $ 1.020 \$ $ & $99.48 \%$& $ 1.749 \$ $ & $99.83 \%$\\

\hline 

\end{tabular}
\end{adjustbox}
% \end{center}
\end{table}

\begin{figure}%
    \centering
    \subfloat[Number of Receipts]{{\includegraphics[width=6cm]{Receipts} }}%
    \qquad
    \subfloat[Cost in USD]{{\includegraphics[width=6cm]{cost} }}%
    \qquad
     \subfloat[Data Percentage Received]{{\includegraphics[width=6cm]{Data} }}%
    \caption{The cost, the number of receipts and the percentage of the data received for three different gas prices}%
    \label{2FigEvaluation}%
\end{figure}

 In figure \ref{2FigEvaluation} (a), the smart contract invocations represented by number of receipts have been increasing for lower consumer reputation and gradually decreased for high reputation. The cost of these invocations have been represented in figure \ref{2FigEvaluation} (b).
%  Note: The cost represented in the figure \ref{2FigEvaluation} is $C_{cost}$.
 
% It is worth to mention that in terms of data producer, he checks the consumer reputation in order to approve a consumer request. The increase in consumer cost by increasing the gas price is not necessary for a producer to accept consumer request and initiate a new trade. While it could be a vital factor for a consumer to increase his reputation faster if and only if the current trade ends successfully. It depends on a producer decision to take this risk and get involve in this trade.


It is worth to mention that the maximum data could be delivered for $C_{rep} = 1 $ for $GUP$ $1.6 Gwei$, $4.2 Gwei$ and $7.2 Gwei$ are $99.10 \%$, $99.48 \%$ and $99.83 \%$, respectively. If we assuming that the minimum time for a transaction to be confirmed is about second, so that the maximum number of messages could be delivered to a consumer is $ \leq (ETM - MR * BN)$. The overhead of processing the receipts -represented as $RT$ - is included in $TATI$ which reduce streaming the data for $RT * BN$.

Although the number of receipts as shown in figure\ref{2FigEvaluation} (a) are nearly the same which lead to the equal cost for receipts as shown in figure\ref{2FigEvaluation} (b), the data percentage received are more than for a higher reputation consumer, figure\ref{2FigEvaluation} (c)

% It is worth to mention that in terms of consumer cost as shown in table \ref{CostTable}, many consumers would prefer to pay the lowest $GUP$ possible because the difference between the cost for a 0.1 consumer reputation and 1.0 is only 9.05 \$ and it is not very high value for one day trade. But In terms of data amount received, figure \ref{2FigEvaluation} (c) has shown that the former has received only about 6.25 \% from the estimated data delivered while the latter has received about 40.10 \% and in real-time data trades, the amount of trade received is the core factor for the consumer, otherwise, the data will lose its value if it is received after long time of its generation. 


\section{CONCLUSIONS} \label{conclusion}

This paper introduces a decentralized marketplace for trading brokered IoT data, and leverage the model with blockchain technology as distributed, immutable and public records. It is structured into two entities: (1) the standard IoT brokered data protocol and on the top and (2) the receipt protocol (as an automated smart contracts on the blockchain network). Every trade has a trade contract signed by both participants and then kept in the blockchain. A TA is a tuple of all trade details between a producer and a consumer.

While the approach is a receipt-based, the real-time data stream is divided into batches which is a logarithmic function of a consumer honesty in the marketplace. As shown, the expenses of having a trade on the blockchain has been categorized into four; registration fees, offering fees, setup fees and the receipt fees. 
There is a inverse proportion between a consumer reputation and an receipt cost. A $f(C_{rep})$ is a directly proportional to $BS$. The evaluation shows that the consumer with high reputation get a small number of batches and therefore less number of smart contract interaction to send receipts. A consumer with low reputation get high number of batches and therefore more gas consumed to interact with the smart contract to send the receipts. 

% \addtolength{\textheight}{-12cm}   % This command serves to balance the column lengths
%                                   % on the last page of the document manually. It shortens
%                                   % the textheight of the last page by a suitable amount.
%                                   % This command does not take effect until the next page
%                                   % so it should come on the page before the last. Make
%                                   % sure that you do not shorten the textheight too much.

% %%%%%%%%%%%%%%%%%%%%%%%%%%%%%%%%%%%%%%%%%%%%%%%%%%%%%%%%%%%%%%%%%%%%%%%%%%%%%%%%



% %%%%%%%%%%%%%%%%%%%%%%%%%%%%%%%%%%%%%%%%%%%%%%%%%%%%%%%%%%%%%%%%%%%%%%%%%%%%%%%%



% %%%%%%%%%%%%%%%%%%%%%%%%%%%%%%%%%%%%%%%%%%%%%%%%%%%%%%%%%%%%%%%%%%%%%%%%%%%%%%%%
% % \section*{APPENDIX}

% % Appendixes should appear before the acknowledgment.

\section*{ACKNOWLEDGMENT}

This research has been funded by Umm AlQura University, Makkah, The Kingdom of Saudi Arabia.



%%%%%%%%%%%%%%%%%%%%%%%%%%%%%%%%%%%%%%%%%%%%%%%%%%%%%%%%%%%%%%%%%%%%%%%%%%%%%%%%
\bibliographystyle{unsrt}
\bibliography{sample,iot-conf}

\end{document}
